\documentclass[fontsize=12pt]{scrartcl} 
\usepackage[T1]{fontenc} 
\usepackage{marvosym}
\usepackage[spanish]{babel}
\usepackage{amsmath,amsfonts,amsthm, amssymb, mathrsfs} 
\usepackage{hieroglf}
\usepackage{bclogo}
\usepackage{cancel}
\providecommand{\abs}[1]{\lvert#1\rvert}
\providecommand{\norm}[1]{\lVert#1\rVert}
\usepackage{fancyhdr} 
\pagestyle{fancyplain} 
\fancyhead{} 
\fancyfoot[L]{} 
\fancyfoot[C]{} 
\fancyfoot[R]{\thepage}
\renewcommand{\headrulewidth}{0pt}
\renewcommand{\footrulewidth}{0pt}
\setlength{\headheight}{13.6pt} 
\usepackage[shortlabels]{enumitem} 
\newtheorem{teo}{Teorema}
\newtheorem{axioma}{Axioma}
\newtheorem{defi}{Definici\'on}
\newtheorem{ejemplo}{Ejemplo}
\newtheorem{coro}{Corolario}
\newtheorem{herram}{Herramienta}
\newtheorem{obs}{Observaci\'on}
\newtheorem{prop}{Proposici\'on}
\newtheorem{conv}{Convenci\'on}


%---------------------------------------------------------------------------------------
%        MACROS
%---------------------------------------------------------------------------------------
\newcommand{\cjto}[2]{\left\{ #1 \colon \ #2 \right\}}
\newcommand{\gen}[1]{\left< #1 \right>}
\newcommand{\pols}[2]{\mathbb{#1}[#2]}
\newcommand{\restr}[2]{#1\upharpoonright_{#2}}
%---------------------------------------------------------------------------------------

\setlength\parindent{0pt} 

%----------------------------------------------------------------------------------------
%	TITULO
%----------------------------------------------------------------------------------------

\newcommand{\horrule}[1]{\rule{\linewidth}{#1}}

\title{	
	\normalfont \normalsize 
	\textsc{UNAM, Facultad de Ciencias} \\ [25pt] 
	\horrule{0.5pt} \\[0.4cm] 
	\huge TAREA 1. N\'umeros enteros $\mathbb{Z}$ y Divisibilidad.\\ 
	\horrule{2pt} \\[0.5cm] 
}

\author{Daniela Ter\'an} 


\date{\normalsize\today} 



%----------------------------------------------------------------------------------------

\begin{document}
	
	
	
	
\maketitle
\begin{center}
		\noindent\rule{\textwidth}{1pt}
%		% Poner el nombre de los alumnos que componen el equipo 
%		% empezando por apellido paterno.
		Integrantes del equipo:\\
		Flores Morán Julieta Melina\\
%		ALUMNO 2\\
%		ALUMNO 3\\
%		ALUMNO 4		
%		
		\noindent\rule{\textwidth}{1pt}  
\end{center}

	Resuelva los ejercicios que se enlistan a continuaci\'on, sea claro y formal 
	en su proceder. El uso de s\'imbolos l\'ogicos queda prohibido.
		
	Indique si las siguientes afirmaciones son verdaderas o falsas. En caso de 
	ser ciertas de una demostraci\'on, en caso contrario un contra ejemplo.
	
	\begin{enumerate}
		\item ({\bf 0.25 puntos}) 		
		Considere $a, b, c \in \mathbb{Z}$ tres n\'umeros enteros arbitrarios.
		Demuestre o brinde alg\'un contraejemplo de lo siguiente:
		
		$1 \mid a$.

                % 2 -- J
		\item ({\bf 0.25 puntos})
		  Si $a \mid b$, entonces $a \mid -b$. \\
                  Esta afirmación es cierta, por lo que daremos una demostración.
		Tenemos como hipótesis que $a \mid b$, por lo tanto, sabemos que $b = a q_1$ con $q_1 \in \mathbb{Z} $.
                Ya que $q_1 \in \mathbb{Z}$, entonces podemos saber que si  $q_2 = q_1 \cdot -1 = -q_1 $ entonces $q_2\in \mathbb{Z}$.\\
                Ahora podemos ver que: \\
                \begin{equation*}
                  \begin{split}
                    a \cdot q_2 &= a \cdot (q_1 \cdot -1) \\
                    &= (a \cdot q_1) \cdot -1 \\
                    &= b \cdot -1 \\
                    &= -b 
                  \end{split}
                \end{equation*}
                
                Por lo tanto, $-b = a \cdot q_2$ con $q_2 \in \mathbb{Z}$ y entonces $a \mid -b$
		\item ({\bf 0.25 puntos})
		$a \mid 0$.

                %4
		\item ({\bf 0.25 puntos})
	
                %5 -- J
		\item ({\bf 0.25 puntos})
		  Si $a \mid b$, entonces $a \leq b$. \\
                  Esta afrimación es falsa, por lo que daremos un contraejemplo: \\
                  Sea $a = 2$ y $b= -10$, $-10 = 2 \cdot -5$ por lo que
                  $2 \mid -10$ pero $2 > -10$ entonces $a > b$ por lo que no se cumple que  $a \leq b$.
		
		\item ({\bf 0.25 puntos})
		Si $a \leq b$, entonces $a\mid b$.
		
		\item ({\bf 0.25 puntos})
		Si $a \mid b$ y $b \mid c$, entonces $a \mid c$.

                %8 -- J
		\item ({\bf 0.25 puntos})
		  Si $a \nmid b$, entonces $b \nmid a$. \\
                  Esta afirmación es falsa, por lo que daremos un contraejemplo: \\
                  Sea $a=25$ y  $b=5$. Podemos verificar que aplicando el algoritmo de la división en a y b, vemos que $5 = 25 \cdot 0 + 5 = a \cdot 0 +5 $. Por lo que $a \nmid b$ pero $25 = 5 * 5 +0$ entonces $b \mid a$ y no se cumple que  $b \nmid a$ .
		
		\item ({\bf 0.25 puntos})
		El $0_{\mathbb{Z}}$ no es par ni impar.
		
		\item ({\bf 0.25 puntos})
		El residuo es cero cuando un n\'umero entero par es divido por el $2$.
		
		\item ({\bf 0.25 puntos})
		Si $a^{2} = b^{2}$, entonces $a = b$.
		
		\item ({\bf 0.25 puntos})
		Si $a \mid (b + c)$, entonces $a \mid b$ y $a \mid c$.
		
		\item ({\bf 0.25 puntos})
		Si $a \mid bc$, entonces $a \mid b$ y $a \mid c$.

                % 14 -- J
		\item ({\bf 0.25 puntos})
		Un entero positivo no primo es un n\'umero compuesto.
		
		\item ({\bf 0.25 puntos})
		Un entero positivo no compuesto es un n\'umero primo.
		
		\item ({\bf 0.25 puntos})
		Todo n\'umero primo es impar.

                %17 -- J
		\item ({\bf 0.25 puntos})
		 No hay primos mayores que un googolplex. \\
                 Esta afirmación no es verdadera, daremos por lo tanto un número primo mayor que un gogolplex.
                 Tomemos el conjunto \\$P = \{p ~ | \text {p es un número primo entre 0 y un googloplex} \}$.
                 Ahora, tomemos
                 $m =  \prod_{n \in P} n  + 1$ donde m es el el producto de todos los números en el conjunto P más uno.
                 Aquí aseguramos que m es mayor que un googolplex y además que ninguno de los números primos menores divide a m, ya que al aplicar el algoritmo de la división entre m y cualquier elemento de P el residuo que queda es igual a 1. Ya que m no se puede dividir entre ningun factor primo más pequeño, entonces hay dos casos: \\
                 Caso 1:
                 m es primo, en este caso, m es un primo mayor que un googleplex.
                 Caso 2: m no es primo:
                 En este caso, como m no tiene como factores a ningún primo del 0 hasta el googleplex, debe tener otro factor primo que es mayor que un googleplex.
                Entonces existe un número primo mayor que un googleplex.
		
		\item ({\bf 0.25 puntos})
		Si $p$ es un primo, entonces $ p + 2 $ es un primo.
		
		\item ({\bf 0.25 puntos})
		Si $ p $ es un primo, entonces $ p^{2} + 1 $ es un primo.

                %20 -- J
		\item ({\bf 0.25 puntos})
		Hay un n\'umero infinito de primos. \\
	        Esta afirmación es verdadera, por lo que lo demoestraremos por contradicción. \\
                Supongamos, para buscar una contradicción, que el conjunto de números primos es finito y que consiste de exactamente los $k$ números primos $p_1, p_2, \cdots, p_k$. Consideremos el número: $m = p_1 \cdot p_2 \cdots \cdot p_k + 1$
El anterior número no es divisible por ninguno de los primos
pues  al aplicar el algoritmo de la división con m y cualquier número primo de $p_1 $ a $p_k$, el residuo que queda es igual a 1. Por el teorema fundamental de la aritmética que dice que cualquier número entero es producto de números primos, entonces m debe tener un divisor primo $p$ diferente de $p_1, p_2, \cdot, p_k$. Esto es una contradicción, pues supusimos que sólo existían los primos $p_1, p_2, \cdot, p_k$. Como está contradicción viene de suponer que existen finitos números primos, entonces podemos concluir que no es así, y hay un número de primos infinitos.

                
		\item ({\bf 0.25 puntos})
		Hay un n\'umero infinito de n\'umeros compuestos.
		
		\item ({\bf 0.25 puntos})
		Si $ p $ es un primo tal que $ p \mid ab $, entonces $ p \mid  a $ 
		o $ p \mid b $.
		
		\item ({\bf 0.25 puntos})
		Hay primos de la forma $n! + 1 $.
		
	
	Demuestre que:
				
		\item ({\bf 0.5 puntos})
		El producto de cualesquiera dos enteros consecutivos es par.
		
		\item ({\bf 0.5 puntos})
		Cualquier entero impar es de la forma $4k + 1$ o $4k + 3$.

                %26 -- J
		\item ({\bf 0.5 puntos}) 
	          $2^{4n} + 3n - 1$ es divisible por $9$  con $n \in \mathbb{N}$ \\
                  Lo demostraremos por inducción sobre n. \\
                  Veremos que con $n \in \mathbb{N}$, $9 \mid 2^{4n} + 3n - 1$. Lo que significa que, exite una $q \in \mathbb{Z} $ que cumple que $2^{4n} + 3n - 1 = 9 \cdot q$ \\
                  \textbf {Paso base:} n = 0 \\
                   \begin{equation*}
                  \begin{split}
                    2^{4n} + 3n - 1 &= 2^{4\cdot 0} + 3 \cdot 0 - 1 \\
                    &= 2^{0} + 0 - 1 \\
                    &= 1 -1 \\
                    &= 0 \\
                    &= 0 \cdot 9
                  \end{split}
                   \end{equation*}
                   Ya que $0 \in \mathbb{Z}$ y $0 = 9 \cdot 0$, entonces $9 \mid 0$.

                   \textbf{Hipótesis de inducción:} \\
                   Para $n \in \mathbb{Z}$ , $2^{4n} + 3n - 1$ es divisible entre 9, es decir, existe $q \in \mathbb{Z} $ tal que $2^{4n} + 3n - 1 = 9 \cdot q$ por lo que $9 \mid 2^{4n} + 3n - 1 $. \\
                   \textbf{Paso inductivo:} \\
                   Probaremos que existe una $q_2 \in \mathbb{Z}$ tal que $2^{4(n+1)} + 3(n+1) - 1 = 9 \cdot q_2$.
                   \begin{equation*}
                  \begin{split}
                    2^{4(n+1)} + 3(n+1) - 1 &= 2^{4n+4} + 3n+21 \\
                    &=  2^{4n} \cdot 2^{4} + 3n+ 2 \\
                    &=  2^{4n} \cdot 16 + 3n+ 2  \\
                    &= 2^{4n} \cdot 16 +48 n -48n +3n -16 + 16 +2  \\
                    &= 2^{4n} \cdot 16 +3n \cdot 16 +16\cdot (-1 )-48n +3n + 16 +2  \\
                    &= 16 ( 2^{4n}  + 3n-1) -48n +3n + 16 +2  \\
                    &= 16 (2^{4n}  + 3n-1) -45n+18 \\
                    &= 16 (9q) -45n+18 \\
                    &= 9 (16q) - (9)(5)n+(9)(2)   \\
                    &=  9 (16q - 5n +2)  \\
                    &=  9 \cdot q_2  \\
                  \end{split}
                   \end{equation*}

                   Vemos que $q_2 = 16q - 5n +2 \in \mathbb{Z}$. Por lo que  $9 \mid 2^{4n} + 3n - 1 $.

Por lo tanto, para cualquier $n \in \mathbb{Z}$, $2^{4n} + 3n - 1$ es divisible por $9$.
		
		\item ({\bf 1 punto}) Encuentre los siguiente dos elementos de cada 
		sucesi\'on y de una descripci\'on recursiva de la misma.
			\begin{enumerate}
				\item $1, 3, 6, 10, 15, \dots$
				\item $1, 4, 10, 20, 35, \dots$
			\end{enumerate}
		
		\item ({\bf 1 punto}) Encuentre los siguientes dos elementos de 
		cada sucesi\'on y obtenga una f\'ormula para la $n$-\'esima 
		posici\'on. Demuestre que la f\'ormula es v\'alida.
			\begin{enumerate}
			\item $\begin{array}[t]{r@{}l}
				1 			&= 	1	\\
				1 + 4 		&= 	5	\\
				1 + 4 + 9 		&= 	14	\\
				1 + 4 + 9 + 16 	&= 	30
			\end{array}$
			\item $\begin{array}[t]{r@{}l}
				1 + 0 \cdot 1 	&= 	\ 1 	\\
				1 + 1 \cdot 3 	&= 	\ 4 	\\
				1 + 2 \cdot 4 	&= 	\ 9 	\\
				1 + 3 \cdot 5 	&= 	16
			\end{array}$
			\end{enumerate}

                %29 -- J
		      \item ({\bf 0.5 puntos}) Supongamos que $p$ y $q$ son n\'umeros
                        
		        primos cuya diferencia es de tres unidades. Demuestre que $p = 5$. \\
                        Ya que la diferencia entre p y q es de 3 unidades,$p-q=3$, emtonces $q=p-3$. Tenemos dos posibilidades para q.
                        \\
                        \textbf{Caso 1:}q es par\\
                        Como q es primo y el único número primo par es 2, entonces q=2 y si  $2=p-3$, entonces $p=5$.\\
                        \textbf{Caso 2:} q es impar: \\
                        Que q sea impar significa que q es de la forma $q=2k+1$ con $k \in \mathbb{Z}$. Así despejamos de $q=p-3$
                        \begin{equation*}
                  \begin{split}
                    2k+1 = p-3\\
                    2k +4 = p\\
                    2(k+2) =p \\
                  \end{split}
                   \end{equation*}
                        Por lo tanto vemos que p es par, ya que $p=1 \cdot (k+2)$. Pero por hiótesis p debe ser primo y el único primo par es 2. Pero veamos que esto no es posible pues si p=2, entonces  de despejar q=p-3 obtenemos $q=2-3=-1$ y ya que q es primo debe ser positivo. Así vemos que este caso no puede suceder.\\
                        Así el único caso posible es el 1, y por lo tanto los únicos números primos cuya diferencia es de 3 unidades son 2 y 5.
                        
                        
		
		\item ({\bf 0.5 puntos}) Si $n$ es un n\'umero compuesto, entonces 
		$2^{n} - 1$ tambi\'en es un n\'umero compuesto.
		

	
		\item ({\bf 0.5 puntos}) Haciendo uso del algoritmo de Euclides encuentre 
		el m.c.d. de los siguientes pares de n\'umeros y luego escr\'ibalo 
		como una combinaci\'on lineal de ellos.
			\begin{enumerate}
				\item $a = -121$ ; $b = 33$.
				\item $a = 543$ ; $b = -241 $.
				\item $a = 78696$ ; $b = 19332$.
				\item $a = -216$ ; $b = 64110$.
				\item $a = 12$ ; $b = -36$.
			\end{enumerate}

                 %32 -- J
	   	 \item ({\bf 0.5 puntos})  Demuestra que dados $a, b \in \mathbb{Z}-\{0\}$ 
		se tiene que $\left( \frac{a}{(a,b)}, \frac{b}{(a,b)} \right ) = 1$.

		\item ({\bf 0.5 puntos}) Sean $a, b, c \in \mathbb{Z}-\{0\}$. Demuestre que 
		$\left[ca, cb\right] = \lvert c \rvert \left[a,b\right]$.
		
		\item ({\bf 0.5 puntos}) Sean $a, b \in \mathbb{Z} - \{0\}$ tales que 
		$\left( a, b \right) = 1$. Demuestre que 
		\[
				\left[a,b\right] = \lvert a \rvert \cdot \lvert b \rvert.
		\]
	\end{enumerate}
	
\end{document}
