\documentclass[fontsize=12pt]{scrartcl} 
\usepackage[T1]{fontenc} 
\usepackage{marvosym}
\usepackage[spanish]{babel}
\usepackage{amsmath,amsfonts,amsthm, amssymb, mathrsfs} 
\usepackage{hieroglf}
\usepackage{bclogo}
\usepackage{cancel}
\providecommand{\abs}[1]{\lvert#1\rvert}
\providecommand{\norm}[1]{\lVert#1\rVert}
\usepackage{fancyhdr} 
\pagestyle{fancyplain} 
\fancyhead{} 
\fancyfoot[L]{} 
\fancyfoot[C]{} 
\fancyfoot[R]{\thepage}
\renewcommand{\headrulewidth}{0pt}
\renewcommand{\footrulewidth}{0pt}
\setlength{\headheight}{13.6pt} 
\usepackage[shortlabels]{enumitem} 
\newtheorem{teo}{Teorema}
\newtheorem{axioma}{Axioma}
\newtheorem{defi}{Definici\'on}
\newtheorem{ejemplo}{Ejemplo}
\newtheorem{coro}{Corolario}
\newtheorem{herram}{Herramienta}
\newtheorem{obs}{Observaci\'on}
\newtheorem{prop}{Proposici\'on}
\newtheorem{conv}{Convenci\'on}


%---------------------------------------------------------------------------------------
%        MACROS
%---------------------------------------------------------------------------------------
\newcommand{\cjto}[2]{\left\{ #1 \colon \ #2 \right\}}
\newcommand{\gen}[1]{\left< #1 \right>}
\newcommand{\pols}[2]{\mathbb{#1}[#2]}
\newcommand{\restr}[2]{#1\upharpoonright_{#2}}
%---------------------------------------------------------------------------------------

\setlength\parindent{0pt} 

%----------------------------------------------------------------------------------------
%	TITULO
%----------------------------------------------------------------------------------------

\newcommand{\horrule}[1]{\rule{\linewidth}{#1}}

\title{	
	\normalfont \normalsize 
	\textsc{UNAM, Facultad de Ciencias} \\ [25pt] 
	\horrule{0.5pt} \\[0.4cm] 
	\huge TAREA 1. N\'umeros enteros $\mathbb{Z}$ y Divisibilidad.\\ 
	\horrule{2pt} \\[0.5cm] 
}

\author{Daniela Ter\'an} 


\date{\normalsize\today} 



%----------------------------------------------------------------------------------------

\begin{document}
	
	
	
	
\maketitle
\begin{center}
		\noindent\rule{\textwidth}{1pt}
%		% Poner el nombre de los alumnos que componen el equipo 
%		% empezando por apellido paterno.
		Integrantes del equipo:\\
		ALUMNO 1\\
%		ALUMNO 2\\
%		ALUMNO 3\\
%		ALUMNO 4		
%		
		\noindent\rule{\textwidth}{1pt}  
\end{center}

	Resuelva los ejercicios que se enlistan a continuaci\'on, sea claro y formal 
	en su proceder. El uso de s\'imbolos l\'ogicos queda prohibido.
		
	Indique si las siguientes afirmaciones son verdaderas o falsas. En caso de 
	ser ciertas de una demostraci\'on, en caso contrario un contra ejemplo.
	
	\begin{enumerate}
		\item ({\bf 0.25 puntos}) 		
		Considere $a, b, c \in \mathbb{Z}$ tres n\'umeros enteros arbitrarios.
		Demuestre o brinde alg\'un contraejemplo de lo siguiente:
		
		$1 \mid a$.
		
		\item ({\bf 0.25 puntos})
		Si $a \mid b$, entonces $a \mid -b$.
		
		\item ({\bf 0.25 puntos})
		$a \mid 0$.
		
		\item ({\bf 0.25 puntos})
		Si $a \mid b$ y $b \mid a$, entonces $a = b$.
		
		\item ({\bf 0.25 puntos})
		Si $a \mid b$, entonces $a \leq b$.
		
		\item ({\bf 0.25 puntos})
		Si $a \leq b$, entonces $a\mid b$.
		
		\item ({\bf 0.25 puntos})
		Si $a \mid b$ y $b \mid c$, entonces $a \mid c$.
		
		\item ({\bf 0.25 puntos})
		Si $a \nmid b$, entonces $b \nmid a$.
		
		\item ({\bf 0.25 puntos})
		El $0_{\mathbb{Z}}$ no es par ni impar.
		
		\item ({\bf 0.25 puntos})
		El residuo es cero cuando un n\'umero entero par es divido por el $2$.
		
		\item ({\bf 0.25 puntos})
		Si $a^{2} = b^{2}$, entonces $a = b$.
		
		\item ({\bf 0.25 puntos})
		Si $a \mid (b + c)$, entonces $a \mid b$ y $a \mid c$.
		
		\item ({\bf 0.25 puntos})
		Si $a \mid bc$, entonces $a \mid b$ y $a \mid c$.
		
		\item ({\bf 0.25 puntos})
		Un entero positivo no primo es un n\'umero compuesto.
		
		\item ({\bf 0.25 puntos})
		Un entero positivo no compuesto es un n\'umero primo.
		
		\item ({\bf 0.25 puntos})
		Todo n\'umero primo es impar.
		
		\item ({\bf 0.25 puntos})
		No hay primos mayores que un googolplex.
		
		\item ({\bf 0.25 puntos})
		Si $p$ es un primo, entonces $ p + 2 $ es un primo.
		
		\item ({\bf 0.25 puntos})
		Si $ p $ es un primo, entonces $ p^{2} + 1 $ es un primo.
		
		\item ({\bf 0.25 puntos})
		Hay un n\'umero infinito de primos.
		
		\item ({\bf 0.25 puntos})
		Hay un n\'umero infinito de n\'umeros compuestos.
		
		\item ({\bf 0.25 puntos})
		Si $ p $ es un primo tal que $ p \mid ab $, entonces $ p \mid  a $ 
		o $ p \mid b $.
		
		\item ({\bf 0.25 puntos})
		Hay primos de la forma $n! + 1 $.
		
	
	Demuestre que:
				
		\item ({\bf 0.5 puntos})
		El producto de cualesquiera dos enteros consecutivos es par.
		
		\item ({\bf 0.5 puntos})
		Cualquier entero impar es de la forma $4k + 1$ o $4k + 3$.
		
		\item ({\bf 0.5 puntos}) 
		$2^{4n} + 3n - 1$ es divisible por $9$ con $n \in \mathbb{N}$ 
		
		\item ({\bf 1 punto}) Encuentre los siguiente dos elementos de cada 
		sucesi\'on y de una descripci\'on recursiva de la misma.
			\begin{enumerate}
				\item $1, 3, 6, 10, 15, \dots$
				\item $1, 4, 10, 20, 35, \dots$
			\end{enumerate}
		
		\item ({\bf 1 punto}) Encuentre los siguientes dos elementos de 
		cada sucesi\'on y obtenga una f\'ormula para la $n$-\'esima 
		posici\'on. Demuestre que la f\'ormula es v\'alida.
			\begin{enumerate}
			\item $\begin{array}[t]{r@{}l}
				1 			&= 	1	\\
				1 + 4 		&= 	5	\\
				1 + 4 + 9 		&= 	14	\\
				1 + 4 + 9 + 16 	&= 	30
			\end{array}$
			\item $\begin{array}[t]{r@{}l}
				1 + 0 \cdot 1 	&= 	\ 1 	\\
				1 + 1 \cdot 3 	&= 	\ 4 	\\
				1 + 2 \cdot 4 	&= 	\ 9 	\\
				1 + 3 \cdot 5 	&= 	16
			\end{array}$
			\end{enumerate}

		\item ({\bf 0.5 puntos}) Supongamos que $p$ y $q$ son n\'umeros 
		primos cuya diferencia es de tres unidades. Demuestre que $p = 5$.
		
		\item ({\bf 0.5 puntos}) Si $n$ es un n\'umero compuesto, entonces 
		$2^{n} - 1$ tambi\'en es un n\'umero compuesto.
		

	
		\item ({\bf 0.5 puntos}) Haciendo uso del algoritmo de Euclides encuentre 
		el m.c.d. de los siguientes pares de n\'umeros y luego escr\'ibalo 
		como una combinaci\'on lineal de ellos.
			\begin{enumerate}
				\item $a = -121$ ; $b = 33$.
				\item $a = 543$ ; $b = -241 $.
				\item $a = 78696$ ; $b = 19332$.
				\item $a = -216$ ; $b = 64110$.
				\item $a = 12$ ; $b = -36$.
			\end{enumerate}
			
	   	 \item ({\bf 0.5 puntos})  Demuestra que dados $a, b \in \mathbb{Z}-\{0\}$ 
		se tiene que $\left( \frac{a}{(a,b)}, \frac{b}{(a,b)} \right ) = 1$.

		\item ({\bf 0.5 puntos}) Sean $a, b, c \in \mathbb{Z}-\{0\}$. Demuestre que 
		$\left[ca, cb\right] = \lvert c \rvert \left[a,b\right]$.
		
		\item ({\bf 0.5 puntos}) Sean $a, b \in \mathbb{Z} - \{0\}$ tales que 
		$\left( a, b \right) = 1$. Demuestre que 
		\[
				\left[a,b\right] = \lvert a \rvert \cdot \lvert b \rvert.
		\]
	\end{enumerate}
	
\end{document}
